      \subsection{Contrast Analyses}
To further explore the impact of climate change terminology on willingness to act, we conducted contrast analyses for both datasets. This analysis aimed to investigate specific exploratory post hoc hypotheses about the differences between groups of terms, categorized by their perceived urgency, scientific connotation, and specificity of threat.\\

 We defined three contrasts:\\

\begin{enumerate}
    \item \textbf{Urgency vs. Neutral:} We hypothesized that urgent/emotional terms (e.g., “climate crisis,” “climate emergency,” “global heating,” “carbon pollution,” and “global boiling”) would elicit stronger support for climate action compared to neutral/scientific terms (e.g., “climate change,” “global warming,” “greenhouse effect,” “carbon emissions,” and “greenhouse gasses”).

    \item \textbf{Scientific vs. Colloquial:} We also hypothesized that there may be a difference between scientific and colloquial language. Specifically, we predicted that scientific terms (e.g., “greenhouse gasses,” “greenhouse effect,” and “carbon emissions”) will be perceived as more credible but less emotionally engaging than colloquial terms (e.g., “climate change,” “global warming,” “climate crisis,” “climate emergency,” and “global boiling”).

    \item \textbf{Specific Threat vs. General Phenomenon:} We hypothesized that terms that specify a direct threat or consequence (e.g., “global heating,” “climate crisis,” “climate emergency,” “carbon pollution,” “global boiling”) will result in higher perceived personal risk compared to terms describing the general phenomenon (e.g., “climate change,” “global warming,” “greenhouse effect,” and “carbon emissions”). 
\end{enumerate}

 For Experiment 1, we found no significant differences between Urgency vs. Neutral (\textit{b} = -2.12, SE = 2.90, \textit{z} = -0.73, \textit{p }= .768, 95\% CI [-8.85, 4.61]), Scientific vs. Colloquial (\textit{b} = 0.01, SE = 2.48, \textit{z} = 0.004, \textit{p} = 1.000, 95\% CI [-5.75, 5.77]), or Specific Threat vs. General Phenomenon (\textit{b} = -3.21, SE = 2.89, \textit{z} = -1.11, \textit{p} = .515, 95\% CI [-9.92, 3.50]). \\

 Similarly, for Experiment 2 we found no significant differences between any contrasts: Urgency vs. Neutral (\textit{b} = 10.63, SE = 7.65, \textit{t }= 1.39, \textit{p} = .344, 95\% CI [-7.09, 28.35]); Scientific vs. Colloquial (\textit{b} = -8.635, SE = 5.95, \textit{t} = -1.45, \textit{p} = .310, 95\% CI [-22.40, 5.13]); Specific Threat vs. General Phenomenon (\textit{b} = 6.72, SE = 6.84, \textit{t} = 0.98, \textit{p} = .600, 95\% CI [-9.11, 22.56]). \\

 The \textit{p}-values for all contrasts were greater than 0.05, indicating that none of the specified contrasts reached statistical significance. 

