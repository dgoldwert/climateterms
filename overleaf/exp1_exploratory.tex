    \subsection{Additional Exploratory Analyses in Experiment All 63 Countries}

\begin{table}[H] 
\caption{Coefficients of a linear mixed effects model with willingness to engage in climate action as the dependent variable, condition (one of 9 terms) as it interacts with perceived competence of scientists (continuous) as the fixed effects, including by-country random effects. }
% latex table generated in R 4.4.1 by xtable 1.8-4 package
% Sat Nov  9 02:31:16 2024
\begin{table}[ht]
\centering
\begin{tabular}{lrrrrrl}
  \hline
Condition & Estimate & \textit{SE} & \textit{df} & \textit{t} & \textit{p} & 95\% CI [LL, UL] \\ 
  \hline
(Intercept) & 27.36 & 3.27 & 3070.33 & 8.37 & \textbf{\textless  .001} & [20.95, 33.75] \\ 
  Carbon Emissions & 0.60 & 4.47 & 3416.83 & 0.13 & .893 & [-8.14, 9.34] \\ 
  Carbon Pollution & 8.27 & 4.48 & 3416.71 & 1.85 & .065 & [-0.49, 17.02] \\ 
  Climate Crisis & 3.35 & 4.53 & 3423.10 & 0.74 & .460 & [-5.51, 12.22] \\ 
  Climate Emergency & -4.22 & 4.61 & 3414.72 & -0.91 & .361 & [-13.24, 4.81] \\ 
  Global Heating & 4.22 & 4.38 & 3417.47 & 0.96 & .335 & [-4.34, 12.78] \\ 
  Global Warming & 4.56 & 4.41 & 3418.10 & 1.03 & .301 & [-4.07, 13.19] \\ 
  Greenhouse Effect & 2.43 & 4.61 & 3424.16 & 0.53 & .598 & [-6.58, 11.44] \\ 
  Greenhouse Gasses & 4.36 & 4.35 & 3413.84 & 1.00 & .317 & [-4.16, 12.87] \\ 
  Trust sci1 1 & 0.67 & 0.04 & 3423.27 & 15.63 & \textbf{\textless  .001} & [0.59, 0.76] \\ 
  Carbon Emissions:Trust sci1 1 & -0.01 & 0.06 & 3415.06 & -0.23 & .815 & [-0.13, 0.10] \\ 
  Carbon Pollution:Trust sci1 1 & -0.12 & 0.06 & 3413.40 & -1.98 & .048 & [-0.24, -0.00] \\ 
  Climate Crisis:Trust sci1 1 & -0.07 & 0.06 & 3422.50 & -1.19 & .236 & [-0.19, 0.05] \\ 
  Climate Emergency:Trust sci1 1 & 0.03 & 0.06 & 3413.51 & 0.56 & .576 & [-0.09, 0.16] \\ 
  Global Heating:Trust sci1 1 & -0.08 & 0.06 & 3415.94 & -1.30 & .192 & [-0.19, 0.04] \\ 
  Global Warming:Trust sci1 1 & -0.08 & 0.06 & 3416.14 & -1.38 & .167 & [-0.20, 0.03] \\ 
  Greenhouse Effect:Trust sci1 1 & -0.08 & 0.06 & 3419.97 & -1.30 & .193 & [-0.20, 0.04] \\ 
  Greenhouse Gasses:Trust sci1 1 & -0.08 & 0.06 & 3412.75 & -1.41 & .158 & [-0.20, 0.03] \\ 
   \hline
\end{tabular}
\end{table}
 
\end{table}
\textit{Note.} We adjusted for multiple comparisons using the Bonferroni correction, thereby using 0.006 as the threshold for statistical significance. Competence of scientists was assessed using a single-item measure “On average, how competent are climate change research scientists?” on a scale from 0 = “not at all” to 100 = “very much so”.

\begin{table}[H] 
\caption{Coefficients of a linear mixed effects model with willingness to engage in climate action as the dependent variable, condition (one of 9 terms) as it interacts with trust in scientific research (continuous) as the fixed effects, including by-country random effects.}
% latex table generated in R 4.4.1 by xtable 1.8-4 package
% Sat Nov  9 23:05:15 2024
\begin{table}[ht]
\centering
\begin{tabular}{lrrrrrl}
  \hline
Condition & Estimate & \textit{SE} & \textit{df} & \textit{t} & \textit{p} & 95\% CI [LL, UL] \\ 
  \hline
(Intercept) & 29.16 & 3.07 & 3014.63 & 9.50 & \textbf{\textless  .001} & [23.15, 35.16] \\ 
  Carbon Emissions & 1.08 & 4.16 & 3416.12 & 0.26 & .795 & [-7.06, 9.22] \\ 
  Carbon Pollution & 4.71 & 4.21 & 3418.71 & 1.12 & .263 & [-3.53, 12.93] \\ 
  Climate Crisis & 6.33 & 4.12 & 3412.58 & 1.54 & .125 & [-1.74, 14.39] \\ 
  Climate Emergency & -4.46 & 4.24 & 3413.28 & -1.05 & .293 & [-12.74, 3.83] \\ 
  Global Heating & 4.87 & 4.10 & 3417.04 & 1.19 & .235 & [-3.15, 12.88] \\ 
  Global Warming & 4.94 & 4.15 & 3417.04 & 1.19 & .234 & [-3.18, 13.05] \\ 
  Greenhouse Effect & 0.49 & 4.35 & 3415.92 & 0.11 & .910 & [-8.03, 9.00] \\ 
  Greenhouse Gasses & 1.30 & 4.25 & 3415.52 & 0.31 & .760 & [-7.00, 9.60] \\ 
  TrustSci & 0.63 & 0.04 & 3422.05 & 16.00 & \textbf{\textless  .001} & [0.56, 0.71] \\ 
  Carbon Emissions:TrustSci & -0.02 & 0.06 & 3414.84 & -0.32 & .752 & [-0.13, 0.09] \\ 
  Carbon Pollution:TrustSci & -0.07 & 0.06 & 3415.63 & -1.26 & .208 & [-0.18, 0.04] \\ 
  Climate Crisis:TrustSci & -0.10 & 0.06 & 3414.02 & -1.88 & .060 & [-0.21, 0.00] \\ 
  Climate Emergency:TrustSci & 0.04 & 0.06 & 3411.87 & 0.77 & .443 & [-0.07, 0.15] \\ 
  Global Heating:TrustSci & -0.08 & 0.05 & 3417.19 & -1.52 & .128 & [-0.19, 0.02] \\ 
  Global Warming:TrustSci & -0.08 & 0.06 & 3414.76 & -1.54 & .124 & [-0.19, 0.02] \\ 
  Greenhouse Effect:TrustSci & -0.05 & 0.06 & 3413.37 & -0.83 & .409 & [-0.16, 0.07] \\ 
  Greenhouse Gasses:TrustSci & -0.05 & 0.06 & 3417.45 & -0.87 & .382 & [-0.16, 0.06] \\ 
   \hline
\end{tabular}
\end{table}
 
\end{table}
\textit{Note.} We adjusted for multiple comparisons using the Bonferroni correction, thereby using 0.006 as the threshold for statistical significance. Trust in scientific research was assessed using a single-item measure “On average, how much do you trust scientific research about climate change?” on a scale from 0 = “not at all” to 100 = “very much so”.

\begin{table}[H] 
\caption{Coefficients of a linear mixed effects model with willingness to engage in climate action as the dependent variable, condition (one of 9 terms) as it interacts with trust in government (continuous) as the fixed effects, including by-country random effects.}
% latex table generated in R 4.4.1 by xtable 1.8-4 package
% Sat Nov  9 23:05:22 2024
\begin{table}[ht]
\centering
\begin{tabular}{lrrrrrl}
  \hline
Condition & Estimate & \textit{SE} & \textit{df} & \textit{t} & \textit{p} & 95\% CI [LL, UL] \\ 
  \hline
(Intercept) & 66.29 & 2.34 & 1091.53 & 28.30 & \textbf{\textless  .001} & [61.71, 70.86] \\ 
  Carbon Emissions & -5.28 & 3.06 & 3410.91 & -1.72 & .085 & [-11.26, 0.71] \\ 
  Carbon Pollution & -2.34 & 2.93 & 3415.29 & -0.80 & .425 & [-8.07, 3.38] \\ 
  Climate Crisis & -5.54 & 2.98 & 3414.15 & -1.86 & .063 & [-11.36, 0.29] \\ 
  Climate Emergency & -4.61 & 2.97 & 3412.86 & -1.56 & .120 & [-10.42, 1.19] \\ 
  Global Heating & -2.79 & 2.94 & 3417.24 & -0.95 & .344 & [-8.55, 2.97] \\ 
  Global Warming & -7.94 & 2.97 & 3417.69 & -2.67 & .008 & [-13.74, -2.13] \\ 
  Greenhouse Effect & -11.62 & 3.11 & 3421.60 & -3.73 & \textbf{\textless  .001} & [-17.71, -5.53] \\ 
  Greenhouse Gasses & -5.64 & 2.91 & 3415.44 & -1.94 & .052 & [-11.34, 0.04] \\ 
  TrustGov & 0.19 & 0.04 & 3434.64 & 5.03 & \textbf{\textless  .001} & [0.12, 0.27] \\ 
  Carbon Emissions:TrustGov & 0.10 & 0.05 & 3408.43 & 1.76 & .078 & [-0.01, 0.20] \\ 
  Carbon Pollution:TrustGov & 0.05 & 0.05 & 3412.49 & 0.95 & .342 & [-0.05, 0.15] \\ 
  Climate Crisis:TrustGov & 0.09 & 0.06 & 3412.22 & 1.67 & .094 & [-0.02, 0.20] \\ 
  Climate Emergency:TrustGov & 0.08 & 0.05 & 3409.97 & 1.46 & .144 & [-0.03, 0.19] \\ 
  Global Heating:TrustGov & 0.03 & 0.05 & 3413.45 & 0.52 & .606 & [-0.08, 0.13] \\ 
  Global Warming:TrustGov & 0.13 & 0.05 & 3412.54 & 2.49 & .013 & [0.03, 0.24] \\ 
  Greenhouse Effect:TrustGov & 0.16 & 0.06 & 3417.92 & 2.98 & \textbf{.003} & [0.06, 0.27] \\ 
  Greenhouse Gasses:TrustGov & 0.09 & 0.05 & 3414.67 & 1.65 & .100 & [-0.02, 0.19] \\ 
   \hline
\end{tabular}
\end{table}
 
\end{table}
\textit{Note.} We adjusted for multiple comparisons using the Bonferroni correction, thereby using 0.006 as the threshold for statistical significance. Trust in government was assessed using a single-item measure “On average, how much do you trust your government?” on a scale from 0 = “not at all” to 100 = “very much so”.

\begin{table}[H] 
\caption{Coefficients of a linear mixed effects model with willingness to engage in climate action as the dependent variable, condition (one of 9 terms) as it interacts with identification as humanitarian (continuous) as the fixed effects, including by-country random effects.}
% latex table generated in R 4.4.1 by xtable 1.8-4 package
% Sat Nov  9 22:54:32 2024
\begin{table}[ht]
\centering
\begin{tabular}{lrrrrrl}
  \hline
Condition & Estimate & \textit{SE} & \textit{df} & \textit{t} & \textit{p} & 95\% CI [LL, UL] \\ 
  \hline
(Intercept) & 31.56 & 4.05 & 3252.19 & 7.79 & \textbf{\textless  .001} & [23.64, 39.47] \\ 
  Carbon Emissions & 3.90 & 5.29 & 3415.84 & 0.74 & .461 & [-6.45, 14.25] \\ 
  Carbon Pollution & -0.60 & 5.47 & 3418.22 & -0.11 & .913 & [-11.31, 10.11] \\ 
  Climate Crisis & -5.26 & 5.43 & 3416.91 & -0.97 & .333 & [-15.89, 5.36] \\ 
  Climate Emergency & -1.81 & 5.47 & 3415.89 & -0.33 & .740 & [-12.51, 8.88] \\ 
  Global Heating & 2.31 & 5.62 & 3419.11 & 0.41 & .682 & [-8.68, 13.31] \\ 
  Global Warming & -4.55 & 5.47 & 3417.11 & -0.83 & .406 & [-15.25, 6.16] \\ 
  Greenhouse Effect & 0.77 & 5.54 & 3412.13 & 0.14 & .889 & [-10.06, 11.62] \\ 
  Greenhouse Gasses & 8.97 & 5.77 & 3423.83 & 1.55 & .120 & [-2.32, 20.25] \\ 
  ID hum 1 & 0.58 & 0.05 & 3432.78 & 11.32 & \textbf{\textless  .001} & [0.48, 0.68] \\ 
  Carbon Emissions:ID hum 1 & -0.06 & 0.07 & 3415.14 & -0.82 & .415 & [-0.19, 0.08] \\ 
  Carbon Pollution:ID hum 1 & 0.00 & 0.07 & 3417.34 & 0.00 & .997 & [-0.14, 0.14] \\ 
  Climate Crisis:ID hum 1 & 0.06 & 0.07 & 3418.47 & 0.81 & .417 & [-0.08, 0.20] \\ 
  Climate Emergency:ID hum 1 & 0.01 & 0.07 & 3416.82 & 0.08 & .940 & [-0.13, 0.14] \\ 
  Global Heating:ID hum 1 & -0.05 & 0.07 & 3416.51 & -0.73 & .468 & [-0.20, 0.09] \\ 
  Global Warming:ID hum 1 & 0.03 & 0.07 & 3415.30 & 0.43 & .667 & [-0.11, 0.17] \\ 
  Greenhouse Effect:ID hum 1 & -0.05 & 0.07 & 3410.75 & -0.67 & .506 & [-0.19, 0.09] \\ 
  Greenhouse Gasses:ID hum 1 & -0.15 & 0.07 & 3424.94 & -2.05 & .040 & [-0.30, -0.01] \\ 
   \hline
\end{tabular}
\end{table}
 
\end{table}
\textit{Note.} We adjusted for multiple comparisons using the Bonferroni correction, thereby using 0.006 as the threshold for statistical significance. Identification as humanitarian was assessed as a single-item measure “To what degree do you see yourself as someone who cares about human welfare?” on a scale from 0 = “not at all” to 100 = “very much so”.

\begin{table}[H] 
\caption{Coefficients of a linear mixed effects model with willingness to engage in climate action as the dependent variable, condition (one of 9 terms) as it interacts with identification as a global citizen (continuous) as the fixed effects, including by-country random effects.}
% latex table generated in R 4.4.1 by xtable 1.8-4 package
% Sat Nov  9 02:31:47 2024
\begin{table}[ht]
\centering
\begin{tabular}{lrrrrrl}
  \hline
Condition & Estimate & \textit{SE} & \textit{df} & \textit{t} & \textit{p} & 95\% CI [LL, UL] \\ 
  \hline
(Intercept) & 37.92 & 3.00 & 3076.03 & 12.64 & \textbf{\textless  .001} & [32.05, 43.79] \\ 
  Carbon Emissions & -1.31 & 4.09 & 3433.25 & -0.32 & .748 & [-9.33, 6.69] \\ 
  Carbon Pollution & -0.97 & 4.15 & 3421.89 & -0.23 & .815 & [-9.08, 7.13] \\ 
  Climate Crisis & -3.64 & 4.22 & 3425.07 & -0.86 & .388 & [-11.90, 4.60] \\ 
  Climate Emergency & -0.88 & 4.14 & 3422.05 & -0.21 & .831 & [-8.98, 7.21] \\ 
  Global Heating & -0.57 & 4.08 & 3419.87 & -0.14 & .889 & [-8.56, 7.41] \\ 
  Global Warming & -0.06 & 4.11 & 3419.50 & -0.02 & .988 & [-8.10, 7.97] \\ 
  Greenhouse Effect & -3.34 & 4.28 & 3422.46 & -0.78 & .434 & [-11.71, 5.02] \\ 
  Greenhouse Gasses & 2.94 & 4.10 & 3428.33 & 0.72 & .472 & [-5.08, 10.95] \\ 
  ID GC 1 & 0.54 & 0.04 & 3435.45 & 13.16 & \textbf{\textless  .001} & [0.46, 0.62] \\ 
  Carbon Emissions:ID GC 1 & -0.01 & 0.06 & 3434.34 & -0.09 & .928 & [-0.11, 0.10] \\ 
  Carbon Pollution:ID GC 1 & -0.00 & 0.06 & 3423.62 & -0.04 & .965 & [-0.11, 0.11] \\ 
  Climate Crisis:ID GC 1 & 0.02 & 0.06 & 3431.36 & 0.31 & .759 & [-0.10, 0.13] \\ 
  Climate Emergency:ID GC 1 & -0.02 & 0.06 & 3424.93 & -0.34 & .737 & [-0.13, 0.09] \\ 
  Global Heating:ID GC 1 & -0.02 & 0.06 & 3423.36 & -0.34 & .732 & [-0.13, 0.09] \\ 
  Global Warming:ID GC 1 & -0.03 & 0.06 & 3420.57 & -0.58 & .561 & [-0.14, 0.08] \\ 
  Greenhouse Effect:ID GC 1 & -0.01 & 0.06 & 3423.13 & -0.09 & .932 & [-0.12, 0.11] \\ 
  Greenhouse Gasses:ID GC 1 & -0.08 & 0.06 & 3434.33 & -1.45 & .146 & [-0.19, 0.03] \\ 
   \hline
\end{tabular}
\end{table}
 
\end{table}
\textit{Note.} We adjusted for multiple comparisons using the Bonferroni correction, thereby using 0.006 as the threshold for statistical significance. Identification as a global citizen was assessed using a single-item measure “To what degree do you think of yourself as a global citizen?” on a scale from 0 = “not at all” to 100 = “very much so”.

\begin{table}[H] 
\caption{Coefficients of a linear mixed effects model with willingness to engage in climate action as the dependent variable, condition (one of 9 terms) as it interacts with environmental identity (continuous) as the fixed effects, including by-country random effects.}
% latex table generated in R 4.4.1 by xtable 1.8-4 package
% Sat Nov  9 22:54:47 2024
\begin{table}[ht]
\centering
\begin{tabular}{lrrrrrl}
  \hline
Condition & Estimate & \textit{SE} & \textit{df} & \textit{t} & \textit{p} & 95\% CI [LL, UL] \\ 
  \hline
(Intercept) & 20.67 & 2.98 & 3230.94 & 6.94 & \textbf{\textless  .001} & [14.85, 26.50] \\ 
  Carbon Emissions & 2.03 & 4.00 & 3422.93 & 0.51 & .612 & [-5.80, 9.86] \\ 
  Carbon Pollution & -0.24 & 4.09 & 3421.22 & -0.06 & .953 & [-8.24, 7.75] \\ 
  Climate Crisis & -1.32 & 4.14 & 3422.27 & -0.32 & .749 & [-9.43, 6.76] \\ 
  Climate Emergency & -3.78 & 4.21 & 3418.62 & -0.90 & .369 & [-12.03, 4.45] \\ 
  Global Heating & -0.84 & 4.09 & 3424.66 & -0.21 & .837 & [-8.86, 7.16] \\ 
  Global Warming & 2.37 & 4.12 & 3416.08 & 0.58 & .565 & [-5.69, 10.42] \\ 
  Greenhouse Effect & -1.22 & 4.39 & 3419.76 & -0.28 & .781 & [-9.80, 7.36] \\ 
  Greenhouse Gasses & -2.51 & 4.17 & 3427.70 & -0.60 & .548 & [-10.68, 5.65] \\ 
  Enviro ID avg & 0.75 & 0.04 & 3440.49 & 19.14 & \textbf{\textless  .001} & [0.67, 0.83] \\ 
  Carbon Emissions:Enviro ID avg & -0.03 & 0.05 & 3425.91 & -0.54 & .591 & [-0.13, 0.08] \\ 
  Carbon Pollution:Enviro ID avg & -0.01 & 0.05 & 3420.57 & -0.18 & .857 & [-0.12, 0.10] \\ 
  Climate Crisis:Enviro ID avg & 0.00 & 0.06 & 3426.62 & 0.03 & .977 & [-0.11, 0.11] \\ 
  Climate Emergency:Enviro ID avg & 0.03 & 0.06 & 3420.67 & 0.55 & .582 & [-0.08, 0.14] \\ 
  Global Heating:Enviro ID avg & -0.01 & 0.05 & 3424.79 & -0.18 & .860 & [-0.12, 0.10] \\ 
  Global Warming:Enviro ID avg & -0.06 & 0.05 & 3415.92 & -1.12 & .262 & [-0.17, 0.05] \\ 
  Greenhouse Effect:Enviro ID avg & -0.03 & 0.06 & 3420.14 & -0.45 & .654 & [-0.14, 0.09] \\ 
  Greenhouse Gasses:Enviro ID avg & 0.00 & 0.06 & 3429.71 & 0.03 & .975 & [-0.11, 0.11] \\ 
   \hline
\end{tabular}
\end{table}
 
\end{table}
\textit{Note.} We adjusted for multiple comparisons using the Bonferroni correction, thereby using 0.006 as the threshold for statistical significance. Environmental identity was assessed by calculating the average of four items: “do you see yourself as someone who cares about the natural environment”, “are you pleased to be someone who cares about the natural environment”, “do you feel strong ties with others who care about the natural environment”, “do you identify with others who care about the natural environment” on a scale from 0 = “not at all” to 100 = “very much so”.

\begin{table}[H] 
\caption{Coefficients of a linear mixed effects model with willingness to engage in climate action as the dependent variable, condition (one of 9 terms) as it interacts with environmental motivation (continuous) as the fixed effects, including by-country random effects.}
% latex table generated in R 4.4.1 by xtable 1.8-4 package
% Sat Nov  9 02:32:03 2024
\begin{table}[ht]
\centering
\begin{tabular}{lrrrrrl}
  \hline
Condition & Estimate & \textit{SE} & \textit{df} & \textit{t} & \textit{p} & 95\% CI [LL, UL] \\ 
  \hline
(Intercept) & 46.33 & 3.09 & 2687.16 & 15.01 & \textbf{\textless  .001} & [40.29, 52.37] \\ 
  Carbon Emissions & 2.36 & 4.22 & 3412.29 & 0.56 & .577 & [-5.91, 10.61] \\ 
  Carbon Pollution & 2.12 & 4.16 & 3413.41 & 0.51 & .611 & [-6.02, 10.27] \\ 
  Climate Crisis & 0.96 & 4.15 & 3414.60 & 0.23 & .817 & [-7.16, 9.07] \\ 
  Climate Emergency & -0.96 & 4.18 & 3407.22 & -0.23 & .818 & [-9.14, 7.21] \\ 
  Global Heating & 3.53 & 4.22 & 3416.85 & 0.84 & .403 & [-4.73, 11.77] \\ 
  Global Warming & -2.25 & 4.22 & 3408.23 & -0.53 & .595 & [-10.51, 6.01] \\ 
  Greenhouse Effect & -0.25 & 4.37 & 3416.15 & -0.06 & .954 & [-8.79, 8.29] \\ 
  Greenhouse Gasses & -0.60 & 4.14 & 3413.63 & -0.14 & .885 & [-8.70, 7.50] \\ 
  Enviro motiv avg & 0.57 & 0.06 & 3425.55 & 10.20 & \textbf{\textless  .001} & [0.46, 0.68] \\ 
  Carbon Emissions:Enviro motiv avg & -0.06 & 0.08 & 3413.41 & -0.74 & .458 & [-0.21, 0.10] \\ 
  Carbon Pollution:Enviro motiv avg & -0.03 & 0.08 & 3411.22 & -0.40 & .689 & [-0.18, 0.12] \\ 
  Climate Crisis:Enviro motiv avg & -0.03 & 0.08 & 3416.95 & -0.40 & .689 & [-0.19, 0.12] \\ 
  Climate Emergency:Enviro motiv avg & -0.00 & 0.08 & 3407.62 & -0.01 & .992 & [-0.15, 0.15] \\ 
  Global Heating:Enviro motiv avg & -0.10 & 0.08 & 3416.30 & -1.25 & .210 & [-0.25, 0.06] \\ 
  Global Warming:Enviro motiv avg & 0.01 & 0.08 & 3406.76 & 0.18 & .857 & [-0.14, 0.17] \\ 
  Greenhouse Effect:Enviro motiv avg & -0.07 & 0.08 & 3413.90 & -0.91 & .362 & [-0.23, 0.08] \\ 
  Greenhouse Gasses:Enviro motiv avg & -0.02 & 0.08 & 3417.33 & -0.29 & .774 & [-0.17, 0.13] \\ 
   \hline
\end{tabular}
\end{table}
 
\end{table}
\textit{Note.} We adjusted for multiple comparisons using the Bonferroni correction, thereby using 0.006 as the threshold for statistical significance. Environmental motivation was assessed by calculating the average of 10 items: “Because of today's politically correct standards, I try to appear pro-environmental.”, “I try to hide my negative thoughts about pro-environmental behavior in order to avoid negative reactions from others.”, “If I acted anti-environmental, I would be concerned that others would be angry with me.”, “I attempt to appear pro-environmental in order to avoid disapproval from others.”, “I try to act pro-environmental because of pressure from others.”, “I attempt to behave pro-environmentally because it is personally important to me.”, “According to my personal values, acting non-environmental is OK.”, “I am personally motivated by my beliefs to be pro-environmental.”, “Because of my personal values, I believe that acting anti-environmental is wrong.”, “Being pro-environmental is important to my self-concept.” on a scale from 0 = “strongly disagree” to 100 = “strongly agree”.

\begin{table}[H] 
\caption{Coefficients of a linear mixed effects model with willingness to engage in climate action as the dependent variable, condition (one of 9 terms) as it interacts with pluralistic ignorance (continuous) as the fixed effects, including by-country random effects.}
% latex table generated in R 4.4.1 by xtable 1.8-4 package
% Sat Nov  9 02:32:11 2024
\begin{table}[ht]
\centering
\begin{tabular}{lrrrrrl}
  \hline
Condition & Estimate & \textit{SE} & \textit{df} & \textit{t} & \textit{p} & 95\% CI [LL, UL] \\ 
  \hline
(Intercept) & 44.78 & 3.40 & 2702.43 & 13.18 & \textbf{\textless  .001} & [38.14, 51.42] \\ 
  Carbon Emissions & 0.94 & 4.54 & 3419.66 & 0.21 & .835 & [-7.93, 9.81] \\ 
  Carbon Pollution & 1.41 & 4.46 & 3411.77 & 0.32 & .753 & [-7.32, 10.12] \\ 
  Climate Crisis & 0.39 & 4.54 & 3408.25 & 0.09 & .932 & [-8.50, 9.27] \\ 
  Climate Emergency & -1.86 & 4.62 & 3414.00 & -0.40 & .687 & [-10.91, 7.17] \\ 
  Global Heating & 8.77 & 4.61 & 3416.23 & 1.90 & .057 & [-0.25, 17.79] \\ 
  Global Warming & 1.43 & 4.53 & 3411.39 & 0.32 & .753 & [-7.43, 10.28] \\ 
  Greenhouse Effect & -4.57 & 4.85 & 3413.15 & -0.94 & .346 & [-14.07, 4.91] \\ 
  Greenhouse Gasses & 7.74 & 4.48 & 3411.98 & 1.73 & .084 & [-1.02, 16.51] \\ 
  PlurIgnoranceItem 1 & 0.49 & 0.05 & 3428.12 & 9.80 & \textbf{\textless  .001} & [0.39, 0.59] \\ 
  Carbon Emissions:PlurIgnoranceItem 1 & -0.02 & 0.07 & 3418.39 & -0.26 & .792 & [-0.15, 0.12] \\ 
  Carbon Pollution:PlurIgnoranceItem 1 & -0.02 & 0.07 & 3410.48 & -0.23 & .818 & [-0.15, 0.12] \\ 
  Climate Crisis:PlurIgnoranceItem 1 & -0.03 & 0.07 & 3410.21 & -0.37 & .715 & [-0.16, 0.11] \\ 
  Climate Emergency:PlurIgnoranceItem 1 & 0.02 & 0.07 & 3414.94 & 0.23 & .815 & [-0.12, 0.16] \\ 
  Global Heating:PlurIgnoranceItem 1 & -0.17 & 0.07 & 3414.16 & -2.39 & .017 & [-0.30, -0.03] \\ 
  Global Warming:PlurIgnoranceItem 1 & -0.05 & 0.07 & 3409.91 & -0.71 & .475 & [-0.18, 0.09] \\ 
  Greenhouse Effect:PlurIgnoranceItem 1 & 0.02 & 0.07 & 3411.76 & 0.27 & .785 & [-0.12, 0.16] \\ 
  Greenhouse Gasses:PlurIgnoranceItem 1 & -0.15 & 0.07 & 3412.11 & -2.19 & .029 & [-0.28, -0.02] \\ 
   \hline
\end{tabular}
\end{table}
 
\end{table}
\textit{Note.} We adjusted for multiple comparisons using the Bonferroni correction, thereby using 0.006 as the threshold for statistical significance. Pluralistic ignorance was assessed using a single-item measure “What percentage of people in your country do you think would agree with the statement ‘Climate change is a global emergency’?”

\begin{table}[H] 
\caption{Coefficients of a linear mixed effects model with willingness to engage in climate action as the dependent variable, condition (one of 9 terms) as it interacts with perceived scientific consensus (continuous) as the fixed effects, including by-country random effects.}
% latex table generated in R 4.4.1 by xtable 1.8-4 package
% Sat Nov  9 22:55:10 2024
\begin{table}[ht]
\centering
\begin{tabular}{lrrrrrl}
  \hline
Condition & Estimate & \textit{SE} & \textit{df} & \textit{t} & \textit{p} & 95\% CI [LL, UL] \\ 
  \hline
(Intercept) & 43.16 & 4.36 & 3172.16 & 9.90 & \textbf{\textless  .001} & [34.63, 51.69] \\ 
  Carbon Emissions & -6.53 & 5.70 & 3412.55 & -1.14 & .252 & [-17.67, 4.63] \\ 
  Carbon Pollution & -1.24 & 5.66 & 3409.14 & -0.22 & .827 & [-12.31, 9.83] \\ 
  Climate Crisis & -5.50 & 5.61 & 3408.40 & -0.98 & .327 & [-16.47, 5.47] \\ 
  Climate Emergency & -12.96 & 5.74 & 3415.99 & -2.26 & .024 & [-24.18, -1.74] \\ 
  Global Heating & -8.37 & 5.80 & 3417.21 & -1.44 & .149 & [-19.71, 2.96] \\ 
  Global Warming & -7.73 & 5.79 & 3408.46 & -1.34 & .182 & [-19.07, 3.59] \\ 
  Greenhouse Effect & -10.95 & 5.91 & 3410.46 & -1.85 & .064 & [-22.52, 0.61] \\ 
  Greenhouse Gasses & -4.38 & 5.74 & 3409.22 & -0.76 & .445 & [-15.60, 6.84] \\ 
  PerceivedSciConsensu 1 & 0.43 & 0.06 & 3416.32 & 7.76 & \textbf{\textless  .001} & [0.32, 0.54] \\ 
  Carbon Emissions:PerceivedSciConsensu 1 & 0.10 & 0.07 & 3415.19 & 1.28 & .202 & [-0.05, 0.24] \\ 
  Carbon Pollution:PerceivedSciConsensu 1 & 0.03 & 0.07 & 3407.50 & 0.39 & .698 & [-0.12, 0.17] \\ 
  Climate Crisis:PerceivedSciConsensu 1 & 0.07 & 0.07 & 3408.61 & 0.89 & .374 & [-0.08, 0.21] \\ 
  Climate Emergency:PerceivedSciConsensu 1 & 0.17 & 0.07 & 3414.48 & 2.27 & .023 & [0.02, 0.32] \\ 
  Global Heating:PerceivedSciConsensu 1 & 0.10 & 0.08 & 3416.09 & 1.30 & .195 & [-0.05, 0.24] \\ 
  Global Warming:PerceivedSciConsensu 1 & 0.08 & 0.08 & 3406.92 & 1.09 & .276 & [-0.06, 0.23] \\ 
  Greenhouse Effect:PerceivedSciConsensu 1 & 0.11 & 0.08 & 3408.56 & 1.42 & .156 & [-0.04, 0.26] \\ 
  Greenhouse Gasses:PerceivedSciConsensu 1 & 0.04 & 0.07 & 3409.43 & 0.59 & .558 & [-0.10, 0.19] \\ 
   \hline
\end{tabular}
\end{table}
 
\end{table}
\textit{Note.} We adjusted for multiple comparisons using the Bonferroni correction, thereby using 0.006 as the threshold for statistical significance. Perceived scientific consensus was assessed using a single-item measure “To the best of your knowledge, what percentage of climate scientists have concluded that human-caused climate change is happening?”

\begin{table}[H] 
\caption{Coefficients of a linear mixed effects model with willingness to engage in climate action as the dependent variable, condition (one of 9 terms) as it interacts with national-level individualism (continuous measure for each country) as the fixed effects, including by-country random effects.}
% latex table generated in R 4.4.1 by xtable 1.8-4 package
% Sat Nov  9 02:32:27 2024
\begin{table}[ht]
\centering
\begin{tabular}{lrrrrrl}
  \hline
Condition & Estimate & \textit{SE} & \textit{df} & \textit{t} & \textit{p} & 95\% CI [LL, UL] \\ 
  \hline
(Intercept) & 84.10 & 3.40 & 454.14 & 24.72 & \textbf{\textless  .001} & [77.46, 90.73] \\ 
  Carbon Emissions & -0.91 & 4.18 & 3431.33 & -0.22 & .828 & [-9.08, 7.29] \\ 
  Carbon Pollution & 0.37 & 3.99 & 3437.09 & 0.09 & .926 & [-7.43, 8.17] \\ 
  Climate Crisis & -2.25 & 4.06 & 3437.11 & -0.55 & .580 & [-10.18, 5.70] \\ 
  Climate Emergency & 1.10 & 4.05 & 3428.88 & 0.27 & .787 & [-6.80, 9.02] \\ 
  Global Heating & -6.20 & 4.06 & 3441.16 & -1.53 & .126 & [-14.13, 1.74] \\ 
  Global Warming & -1.84 & 4.02 & 3439.97 & -0.46 & .648 & [-9.68, 6.03] \\ 
  Greenhouse Effect & -8.43 & 4.15 & 3450.66 & -2.03 & .042 & [-16.53, -0.31] \\ 
  Greenhouse Gasses & -3.83 & 3.98 & 3435.56 & -0.96 & .336 & [-11.60, 3.96] \\ 
  Ind score & -0.18 & 0.06 & 256.62 & -3.14 & \textbf{.002} & [-0.30, -0.07] \\ 
  Carbon Emissions:Ind score & 0.01 & 0.07 & 3423.59 & 0.13 & .900 & [-0.12, 0.14] \\ 
  Carbon Pollution:Ind score & -0.01 & 0.06 & 3427.15 & -0.09 & .927 & [-0.13, 0.12] \\ 
  Climate Crisis:Ind score & 0.00 & 0.06 & 3427.94 & 0.07 & .941 & [-0.12, 0.13] \\ 
  Climate Emergency:Ind score & -0.05 & 0.06 & 3420.10 & -0.70 & .484 & [-0.17, 0.08] \\ 
  Global Heating:Ind score & 0.08 & 0.06 & 3431.99 & 1.20 & .230 & [-0.05, 0.20] \\ 
  Global Warming:Ind score & -0.00 & 0.06 & 3429.39 & -0.05 & .963 & [-0.13, 0.12] \\ 
  Greenhouse Effect:Ind score & 0.09 & 0.07 & 3441.17 & 1.32 & .188 & [-0.04, 0.21] \\ 
  Greenhouse Gasses:Ind score & 0.03 & 0.06 & 3425.36 & 0.48 & .632 & [-0.09, 0.15] \\ 
   \hline
\end{tabular}
\end{table}
 
\end{table}
\textit{Note.} We adjusted for multiple comparisons using the Bonferroni correction, thereby using 0.006 as the threshold for statistical significance.